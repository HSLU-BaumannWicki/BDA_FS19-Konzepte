\chapter{Ausgangslage}
In der Speicherbibliothek in Büron, Luzern ist ein Behälter-Hochregallager erbaut worden. In diesem Hochregallager, werden bis zu 110'000 Behältern verschiedene Exemplare (welche viele mit RFID Tags ausgestattet sind) gelagert. Die Behälter werden manuell von Menschen befüllt und anschliessend wird der Behälter autonom an einen Lagerplatz gefahren. Zeitweise können Exemplare wieder aus den Behältern entnommen werden um diese zu Lesen, Scannen oder einer der Teilnehmenden Bibliotheken zurückzusenden. Während dem Vorgang des Lagerns und Entnehmen der Exemplare werden Menschen für das Befüllen/Entfernen der Behälter verwendet. Dies birgt die Gefahr, dass eine Person aus Versehen ein Exemplar in einen falschen Behälter legt und so das Exemplar nur sehr umständlich wiedergefunden werden kann.