\chapter{Ausgangslage}
In der Speicherbibliothek in Büron, Luzern wurde ein Hochregallager für Behälter erbaut. In diesem Hochregallager, werden bis zu 110'000 Behältern mit verschiedenen Exemplaren (welche viele mit RFID Tags ausgestattet sind) gelagert. Die Behälter werden manuell durch Menschen befüllt und anschliessend autonom an einen Lagerplatz gefahren. Zeitweise können Exemplare wieder aus den Behältern entnommen werden um gelesen, gescannt oder einer der teilnehmenden Bibliotheken gesendet zu werden. Während dem Vorgang des Lagerns und Entnehmen der Exemplare, geschieht das Entnehmen und Befüllen durch Mitarbeiter der Speicherbibliothek. Dies birgt die Gefahr, dass eine Person aus Versehen ein Exemplar in einen falschen Behälter legt und dieses danach nur sehr umständlich wiedergefunden werden kann.
