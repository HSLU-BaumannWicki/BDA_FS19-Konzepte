\chapter{Strategie}

\section{Ziele}
Sollte ein Exemplar verlegt werden, soll dieses vor Einlagerung des betroffenen Behälters bemerkt werden und dieser aus dem Einlagerungsprozess aussortiert oder markiert werden.

\section{Wirkung}
Die so behobene Möglichkeit eines Einlagerns eines falsch befüllten Behälters führt zu einem erhöhten Vertrauen in den Prozess. Das Verlegen eines Exemplars soll so verunmöglicht werden.

\section{Zielgruppe}
Dies führt zu besserem Vertrauen der Mitarbeiter der direkten Zielgruppe Speicherbibliothek in den Prozess und bessere Möglichkeit für den Fokus auf die Hauptarbeitstätigkeit (nicht Kontrolle eines Behälters). Die indirekte Zielgruppe währen demnach alle der Speicherbibliothek angeschlossenen Bibliotheken, sowie deren Kunden, welche Seiten bestimmter Exemplare als eingescannte Datei bestellen oder ausgewählte Exemplare mit einer Voranmeldung vor Ort lesen wollen.

\chapter{Ideenbeschreibung}

Die generelle Idee dieses Konzepts ist eine Station in das Förderband zu integrieren, an der der Behälter durchfährt und an der Stelle dieser gescannt wird und Unstimmigkeiten erkannt werden. Dafür würden verschiedene Orte und Komponenten in Betracht gezogen. Diese sind daher jeweils einzeln in Betracht gezogen worden.

\section{Position}

\section{Identifikation der RFID Tags}

\section{Identifikation des Behälters}

\section{Massnahmen nach Erkennungsprozess}

\chapter{Zeitplan}

\chapter{Finanzierungsplan}

\chapter{Dokumentation und Evaluation}