\chapter{Strategie}

\section{Ziel}
Sollte ein bestimmtes Exeemplar verloren gehen, soll diese automatiesiert im Hochregallager wiedergefunden werden.

\section{Wirkung}
Die so vereinfachte Wiederfindung soll der Speicherbiliothek zu einer erhöhten Sicherheit bezüglich der Einlagerung der Exemplare verhelfen. Es soll bewirkt werden, dass deplatzierte Exemplare ohne grösseren Menschlichen Aufwand wiederbeschafft werden können.

\section{Zielgruppe}
Die Direkte Zielgruppe ist die Speicherbiliothek selbst. Diese könnte somit der Indirekten Zielgruppe eine noch höhere Garantie gewähreleisten.

Die Indirekte Zielgruppe währen demnach alle der Speicherbibliothek angeschlossenen Bibliotheken, sowie weiteren Kunden, welche Exemplarseiten als eingescannte Seiten bestellen oder mit einer Voranmeldung Exemplare vo Ort lesen können.

\chapter{Ideenbeschreibung}

\chapter{Zeitplan}

\chapter{Finanzierungsplan}

\chapter{Dokumentation und Evaluation}